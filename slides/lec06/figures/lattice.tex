\documentclass[tikz]{standalone}

\usetikzlibrary{calc,positioning}

% Create fake \onslide and other commands for standalone picture
\usepackage{xparse}
\NewDocumentCommand{\onslide}{s t+ d<>}{}
\NewDocumentCommand{\only}{d<>}{}
\NewDocumentCommand{\uncover}{d<>}{}
\NewDocumentCommand{\visible}{d<>}{}
\NewDocumentCommand{\invisible}{d<>}{}

\definecolor{ForestGreen}{RGB}{34,139,34}

\begin{document}

%\tikzset{
%    dot/.style={circle, outer sep=0pt}
%}
\tikzset{vector/.style={draw,rounded corners,very thick,#1,fill=#1!15,minimum width=2cm,minimum height=1cm}}
\tikzset{vector/.default={blue}}
\tikzset{arrow/.style={draw,->,>=stealth,very thick,#1}}
\tikzset{arrow/.default={black}}
%\tikzset{font={\fontsize{18pt}{12}\selectfont}}

\tikzset{state/.style={draw,circle,very thick,#1,fill=#1!15,inner sep=0pt,minimum size=1cm}}
\tikzset{state/.default={blue}}

    \begin{tikzpicture}
    \tikzset{font={\fontsize{18pt}{12}\selectfont}}
    \foreach \i/\w in {1/language, 2/is, 3/fun}{
        \node (s\i1) [state] at(2*\i,0) {N};
        \foreach \j/\t in {2/V, 3/A}{
            \pgfmathtruncatemacro\k{\j-1}
            \node (s\i\j) [state] [below= of s\i\k] {\t};
        }
    }
    \node (start) [state] [left=1cm of s12] {};
    \node (end) [state] [right=1cm of s32] {};

    \foreach \i in {2,...,3}{
        \pgfmathtruncatemacro\k{\i-1}
        \foreach \j in {1,...,3}{
            \foreach \t in {1,...,3}{
                \path[arrow={black!50}] (s\k\t) -- (s\i\j) ;
            }
        }
    }
    \foreach \i in {1,...,3}{
        \path[arrow={black!50}] (start) -- (s1\i);
        \path[arrow={black!50}] (s3\i) -- (end);
    }

    \foreach \i/\w in {1/language, 2/is, 3/fun}{
        \node (y\i) [above=0.3cm of s\i1] {$y_\i$};
        \node (x\i) [below=0.3cm of s\i3] {\w};
    }
    \node (ts) at(y1 -| start) {\texttt{START}};
    \node (te) at(y3 -| end) {\texttt{STOP}};

    %\path[arrow={red}] ([xshift=-1ex]end.south) -- ([yshift=1ex]s33.east);
    %\path[arrow={red}] ([yshift=1ex]s33.west) -- ([xshift=1ex]s22.south);
    %\path[arrow={red}] ([yshift=1ex]s22.west) -- ([xshift=1ex]s11.south);
    %\path[arrow={red}] ([xshift=-1ex]s11.south) -- ([yshift=1ex]start.east);

\end{tikzpicture}

\end{document}

