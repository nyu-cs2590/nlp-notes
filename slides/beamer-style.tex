%  beamer style
%\usetheme{Boadilla}
%\useoutertheme{infolines}

% color
\newif\ifstartedinmathmode
\definecolor{beamerblue}{rgb}{0.2,0.2,0.7}
\definecolor{green}{RGB}{34,139,34}
\newcommand<>{\gray}[1]{\textcolor#2{gray}{#1}}
\newcommand<>{\red}[1]{\textcolor#2{red}{#1}}
\newcommand<>{\mred}[1]{{\color#2{red}#1}}
\newcommand<>{\blue}[1]{\textcolor#2{blue}{#1}}
\newcommand<>{\mblue}[1]{{\color#2{blue}#1}}
\newcommand<>{\green}[1]{\textcolor#2{green}{#1}}
\newcommand<>{\mgreen}[1]{{\color#2{green}#1}}
\newcommand<>{\beamerblue}[1]{\textcolor#2{beamerblue}{#1}}
\newcommand<>{\mbeamerblue}[1]{{\color#2{beamerblue}#1}}
\newcommand{\mycite}[1]{{\color{gray}\small #1}}
%\newcommand{\green}[1]{\textcolor{ForestGreen}{#1}}
%\newcommand{\red}[1]{\textcolor{red}{#1}}
%\newcommand{\blue}[1]{\textcolor{blue}{#1}}


%\setbeamercolor{frametitle}{fg=blue}
%\setbeamercolor{title}{fg=black}
\setbeamertemplate{footline}[frame number]
\setbeamertemplate{navigation symbols}{}
\setbeamertemplate{blocks}[rounded][shadow=false]
\addtobeamertemplate{block begin}{}{}
\addtobeamertemplate{block end}{}{\medskip}

% itemize
\newenvironment{wideitemize}{\itemize\addtolength{\itemsep}{10pt}}{\enditemize}
\setbeamertemplate{itemize items}[default]
\setbeamertemplate{enumerate items}[default]
%\setbeamerfont*{itemize/enumerate body}{size=\large}
\setbeamerfont*{itemize/enumerate subbody}{parent=itemize/enumerate body}
\setbeamerfont*{itemize/enumerate subsubbody}{parent=itemize/enumerate body}


% Background
\definecolor{MyBackground}{RGB}{255,253,218}
%% Uncomment this if you want to change the background color to something else
%\setbeamercolor{background canvas}{bg=MyBackground}

%% Change the bg color to adjust your transition slide background color!
\newenvironment{transitionframe}{
  \setbeamercolor{background canvas}{bg=yellow}
  \begin{frame}}{
    \end{frame}
}

\newenvironment<>{spbk}[1]{%
  \begin{actionenv}#2%
      \def\insertblocktitle{#1}%
      \par%
      \mode<presentation>{%
        \setbeamercolor{block title}{fg=normal text.fg, bg=normal text.bg}
       \setbeamercolor{block body}{bg=normal text.bg}
     }%
      \usebeamertemplate{block begin}}
    {\par\usebeamertemplate{block end}\end{actionenv}}
    

% tikz styles
\usepackage{tikz}
\usepackage{xparse}%  For \NewDocumentCommand
\usepackage{calc}%    For the \widthof macro
\usetikzlibrary{shapes.geometric, arrows}
\usetikzlibrary{calc}
\usetikzlibrary{tikzmark}
 \usetikzlibrary{overlay-beamer-styles}
\usetikzlibrary{positioning}
\usepackage{tikz-qtree}



%\tikzstyle{textbox}=[rectangle, rounded corners, text centered, draw=#1, thick, fill=#1!15]
\tikzstyle{textboxMinimal}=[rectangle, text centered, draw=#1, thick]
\tikzstyle{textcircle}=[circle, text centered, draw=#1, thick, fill=#1!15, minimum width=0.8cm, inner sep=0pt]
\tikzstyle{arrow}=[thick, ->, >=stealth]

\tikzset{
	textbox/.style 2 args={rectangle, rounded corners, text centered, draw={#1}, thick, fill={#1!15}, minimum height={#2}},
	textbox/.default={blue}{1cm},
}

\tikzset{
    invisible/.style={opacity=0},
    visible on/.style={alt={#1{}{invisible}}},
    alt/.code args={<#1>#2#3}{%
      \alt<#1>{\pgfkeysalso{#2}}{\pgfkeysalso{#3}} % \pgfkeysalso doesn't change the path
    },
  }

%% draw box
%\newcommand{\tikzmark}[1]{\tikz[overlay,remember picture] \node (#1) {};}

\makeatletter
\NewDocumentCommand{\DrawBox}{s O{}}{%
    \tikz[overlay,remember picture]{
    \IfBooleanTF{#1}{%
        \coordinate (RightPoint) at ($(pic cs:left |- pic cs:right)+(\linewidth-\labelsep-\labelwidth,0.0)$);
    }{%
        \coordinate (RightPoint) at (pic cs:right.east);
    }%
    \draw[red,#2]
      ($(pic cs:left)+(-0.2em,0.9em)$) rectangle
      ($(RightPoint)+(0.2em,-0.3em)$);}
}

\NewDocumentCommand{\DrawBoxWide}{s O{}}{%
    \tikz[overlay,remember picture]{
    \IfBooleanTF{#1}{%
        \coordinate (RightPoint) at ($(pic cs:left |- pic cs:right)+(\linewidth-\labelsep-\labelwidth,0.0)$);
    }{%
        \coordinate (RightPoint) at (pic cs:right.east);
    }%
    \draw[red,#2]
      ($(pic cs:left)+(-\labelwidth,0.9em)$) rectangle
      ($(RightPoint)+(0.2em,-2em)$);}
}
\makeatother

%% comment box
% tikz
\newenvironment{commentbox}[1][blue]
  {\begin{tikzpicture}[remember picture,overlay,
  	expl/.style={draw=#1,fill=#1!20,rounded corners,text width=4cm},
  	arrow/.style={#1,thick,->,>=latex}]
  }
  {\end{tikzpicture}
  }
  
% tikz overlay
\tikzset{
fade/.style={opacity=#1},
fade/.default={1},
fade on/.style={alt=#1{}{fade}},
}

% other styles

% textbox
\newenvironment{filledbox}[2][beamerblue]
  {\begin{tikzpicture}
    \def\boxcolor{#1}
      \node [draw=#1, fill=#1!10, thick, rectangle, rounded corners, inner sep=10pt, inner ysep=10pt, text width=#2] (box)
      \bgroup
  }
  {
      \egroup;
    \end{tikzpicture}
  }

\newenvironment{titledbox}[3][beamerblue]
  {\begin{tikzpicture}
    \def\boxcolor{#1}
    \def\boxname{#3}
    \node [draw=#1, thick, rectangle, rounded corners, inner sep=10pt, inner ysep=10pt, text width=#2] (box)
      \bgroup
  }
  {
      \egroup;
      \node[fill=white,right=10pt] at (box.north west) {\textcolor{\boxcolor}{\boxname}};
    \end{tikzpicture}
  }

% icon
\newcommand{\idea}[1]{
    \begin{tikzpicture}
        \node[anchor=south](bulb){\includegraphics[height=1cm]{figures/lightbulb}};
        \node[right=0cm of bulb, anchor=west, text width=0.9\linewidth]{#1};
    \end{tikzpicture}
}
